\section{Talous}
Lippukunta on säilynyt vakavaraisena läpi tilikauden. Lippukunta sai Urlus-säätiöltä 5000\euro{} avustusta Kyöpelin katon uusimiseen, tämä toteutetaan <joskus>. Lippukunta haki ja sai myös 1500\euro{} rahaa leirikalustonsa uusimiseen, tällä hankittiin uusi puolijoukkueteltta. Varoja koitetaan säästää tulevaisuutta varten, esimerkiksi suojaamaan nousevalta toimitilavuokralta.\\
\\Varainhankintaa toteutettiin perinteisin keinoin. Näihin kuuluivat esimerkiksi adventtikalentereiden myyminen sekä osallistuminen Lions Clubin ja Mellunmäki-seuran tapahtumiin.\\
\\Kyöpelin suhteen varoja kerätään aktiivisesti vuokratuloilla. Vuokratulot käytetään pääsääntöisesti kiinteistön kunnostamiseen ja muihin huoltotoimenpiteisiin. Lippukunta sai myös 5000\euro{} apurahan Urlus-säätiöltä kattoremonttia varten ja noin saman suuruisen summan osana Haapakerttujen viime vuotista lahjakirjaa. Viimeisenä mainittu simma on korvamerkitty Kyöpeliä varten, eikä siihen kosketa ilman lippukunnan hallituksen erillistä päätöstä


